% LaTeX Curriculum Vitae Template
%
% Copyright (C) 2004-2009 Jason Blevins <jrblevin@sdf.lonestar.org>
% http://jblevins.org/projects/cv-template/
%
% You may use use this document as a template to create your own CV
% and you may redistribute the source code freely. No attribution is
% required in any resulting documents. I do ask that you please leave
% this notice and the above URL in the source code if you choose to
% redistribute this file.

\documentclass[letterpaper]{article}

\usepackage{hyperref}
\usepackage{geometry}

% Comment the following lines to use the default Computer Modern font
% instead of the Palatino font provided by the mathpazo package.
% Remove the 'osf' bit if you don't like the old style figures.
\usepackage[T1]{fontenc}
\usepackage[sc,osf]{mathpazo}

% Set your name here
\def\name{Introduction to Political Economy}

% Replace this with a link to your CV if you like, or set it empty
% (as in \def\footerlink{}) to remove the link in the footer:
\def\footerlink{}
% \href{http://www.hectorbahamonde.com}{www.HectorBahamonde.com}

% The following metadata will show up in the PDF properties
\hypersetup{
  colorlinks = true,
  urlcolor = blue,
  pdfauthor = {\name},
  pdfkeywords = {political science, economic development, methods},
  pdftitle = {\name: Curriculum Vitae},
  pdfsubject = {Curriculum Vitae},
  pdfpagemode = UseNone
}

\geometry{
  body={6.5in, 8.5in},
  left=1.0in,
  top=1.25in
}

% Customize page headers
\pagestyle{myheadings}
\markright{{\tiny \name}}
\thispagestyle{empty}

% Custom section fonts
\usepackage{sectsty}
\sectionfont{\rmfamily\mdseries\Large}
\subsectionfont{\rmfamily\mdseries\itshape\large}

% Other possible font commands include:
% \ttfamily for teletype,
% \sffamily for sans serif,
% \bfseries for bold,
% \scshape for small caps,
% \normalsize, \large, \Large, \LARGE sizes.

% Don't indent paragraphs.
\setlength\parindent{0em}

% Make lists without bullets
\renewenvironment{itemize}{
  \begin{list}{}{
    \setlength{\leftmargin}{1.5em}
  }
}{
  \end{list}
}

\begin{document}

% Place name at left
%{\huge \name}

% Alternatively, print name centered and bold:
\centerline{\huge \bf \name}

\vspace{0.25in}

\begin{minipage}{0.45\linewidth}
  Rutgers University, New Brunswick \\
  Political Science Department \\
  Hickman Hall \\
  New Brunswick, NJ 08901\\
  \\
  \\

\end{minipage}
\hspace{4cm}\begin{minipage}{0.45\linewidth}
  \begin{tabular}{ll}
{\bf Last updated}: \today. \\
 {\bf Download last version} \href{https://github.com/hbahamonde/Political-Economy-Intro-UGrad/raw/master/Pol_Econ_Dev_Syllabus_UGRAD.pdf}{here}.
    \\
    \\
    \\
    \\
    \\
    \\
  \end{tabular}
\end{minipage}

\vspace{-5mm}
{\bf Instructor}: H\'ector Bahamonde\\
\texttt{e:}\href{mailto:hector.bahamonde@rutgers.edu}{\texttt{hector.bahamonde@rutgers.edu}}\\
\texttt{w:}\href{http://www.hectorbahamonde.com}{\texttt{www.hectorbahamonde.com}}\\
{\bf Location}: Classroom.\\
{\bf Office Hours}: Make an appointment \href{https://calendly.com/bahamonde/officehours}{\texttt{here}}.\\
{\bf Class Website and Materials}: click \href{https://github.com/hbahamonde/Political-Economy-Intro-UGrad}{\texttt{here}}.

\subsection*{Overview and Objectives}

This {\bf {\color{blue}undergraduate-level course}} is intended as an introduction to political economy, specially, the politics of institutions and long-run development. The papers will draw from political economy, development economics, economic history, fiscal sociology, institutional economics and some times, applied econometrics. However, we will focus on the theoretical discussion rather than the econometrics behind it.



\subsection*{Course Learning Objectives}
 
Upon successful completion of this course, you will be able to:

\begin{itemize}
	\item[$\bullet$] Acquire an understanding of the main CPE theories and topics.
	\item[$\bullet$] Use the comparative method and analysis in the political science literature.
	\item[$\bullet$] Consume critically the CPE/Development literature.
\end{itemize}



\subsection*{Requirements}

In this course we will cover the key concepts and theoretical debates in a large sub-field in comparative politics. Students will be expected to complete the required readings each week, attend the lectures, participate in class discussions and take careful notes. When readings the class materials, you should locate the main argument, strengths, weaknesses, and other issues that are of concern. If there are certain questions or points that you think we should specifically examine in class, mark them down and raise them in our class discussions.

\subsection*{Evaluation}


\begin{itemize}
	\item[$\bullet$] {\bf Two midterm exams}: 25 \%.
	\item[$\bullet$] {\bf Final exam}: 25 \%.
	\item[$\bullet$] {\bf Participation}: 25 \%.
\end{itemize}


\subsection*{Academic Integrity}
In accordance with Rutgers University policy on Academic Integrity, you are expected to fully comply with the school’s policies.  Please see this \href{http://academicintegrity.rutgers.edu}{\texttt{link}}.


\subsection*{Students with Disabilities}
Students with disabilities who require accommodation should review the following statement from the Office of Disability Services \href{https://ods.rutgers.edu/faculty/syllabus}{\texttt{link}}.


\subsection*{Absence from Exams}

Only a note from your college dean stipulating a medical or family emergency will be acceptable as an excuse for missing an exam. If at all possible, I need to be notified before the exam of your inability to take it. Absence from an exam because of travel plans will not be excused.

\subsection*{Cell Phones} 

Make sure your cell phones are turned OFF before entering class.


\subsection*{Schedule}

\begin{enumerate}

\item {\bf Perspectives on Development}
	\begin{itemize}
		\item[$\bullet$] Sachs, J. (2005). \emph{The End of Poverty}. Chapter 3: ``Why Some Countries Fail to Thrive.'' Penguin.
		\item[$\bullet$] Easterly, W. (2006). \emph{The White Man's Burden}. Chapter 1: ``Planners versus Searchers.'' Penguin.
		\item[$\bullet$] Banerjee,A. and E.Duflo.(2011). \emph{Poor Economics}. Chapter 3: ``Low-Hanging Fruit for Better (Global) Health.'' Public Affairs.
	\end{itemize}


\item {\bf Importance of Politics: Example from Africa}
	\begin{itemize}
		\item[$\bullet$] Bates, R. (2008). \emph{When Things Fell Apart}. Chapter 2: ``From Fable to Fact.'' Cambridge.
	\end{itemize}


\item {\bf Origins of Democracy}
	\begin{itemize}
		\item[$\bullet$] Acemoglu, D. and J. Robinson (2006). \emph{Economic Origins of Dictatorship and Democracy}. Chapters 1 and 2: ``Paths of Political Development'' and ``Our Argument.'' Cambridge.
		\item[$\bullet$] Boix, C. (2003). \emph{Democracy and Redistribution}. Chapter 3: ``Historical Evidence.'' Cambridge.
		\item[$\bullet$] Ansell, B. and D. Samuels (2014). \emph{Inequality and Democratization}. Chapters 1 and 2: ``Introduction'' and ``Inequality, Development, and Distribution.'' Cambridge.
		\item[$\bullet$] Dasgupta, A. and D. Ziblatt (2015). ``How Did Britain Democratize? Views from the Sovereign Bond Market.'' \emph{Journal of Economic History}, 75: 1-29. (Skip Section “Results”)
	\end{itemize}


\item {\bf Origins of States}
	\begin{itemize}
		\item[$\bullet$] Boix, C. (2015). \emph{Political Order and Inequality}. Chapter 2: ``Political Order.'' Cambridge.
		\item[$\bullet$] Bates, R. (2010). \emph{Prosperity and Violence}. Chapter 3: ``The Formation of States.''. Norton.
		\item[$\bullet$] Drelichman, M. and H.J. Voth (2014). \emph{Lending to the Borrower from Hell}. Chapter 8: ``Tax, Empire, and the Logic of Spanish Decline.'' Princeton.
		\item[$\bullet$] Dincecco, M. (2015). ``The Rise of Effective States in Europe.'' \emph{Journal of Economic History}, 75: 901-18.
	\end{itemize}


\item {\bf Warfare, State Formation, and Colonialism}
	\begin{itemize}
		\item[$\bullet$] Hoffman, P. (2015). \emph{Why Did Europe Conquer the World?} Chapter 2: ``How the Tournament in Early Modern Europe Mad Conquest Possible.'' Princeton.
		\item[$\bullet$] E. Akyeampong, R. Bates, N. Nunn, and J. Robinson, eds. (2014). \emph{Africa’s Development in Historical Perspective}. Chapter 14: ``The Imperial Peace.'' Cambridge.
	\end{itemize}


\item {\bf Why There May Be No State}
	\begin{itemize}
		\item[$\bullet$] Herbst, J. (2000). \emph{States and Power in Africa}. Chapter 5: ``National Design and the Broadcasting of Power.'' Princeton.
		\item[$\bullet$] Scott, J. (2009). \emph{The Art of Not Being Governed}. Chapter 1: ``Hills, Valleys, and States.'' Yale.
	\end{itemize}


\item {\bf Does Democracy Foster Growth?}
	\begin{itemize}
		\item[$\bullet$] Acemoglu, D. and J. Robinson (2012). \emph{Why Nations Fail}. Chapter 3: ``The Making of Prosperity and Poverty.'' Profile.
		\item[$\bullet$] Rosenthal, J.L. (1992). \emph{Fruits of Revolution}. Chapter 3: ``Institutions and Economic Growth.'' Cambridge.
		\item[$\bullet$] E. Helpman, ed. (2009). \emph{Institutions and Economic Performance}. Chapter 11: ``Making Autocracy Work,'' Besley, T. and M. Kudamatsu. Harvard.
	\end{itemize}



\item {\bf What Can Governments Do?}
	\begin{itemize}
		\item[$\bullet$] Lindert, P. (2004). \emph{Growing Public}. Chapter 5: ``The Rise of Mass Public Schooling Before 1914.'' Cambridge.
		\item[$\bullet$] Goldin, C. and K. Katz (2010). \emph{The Race between Education and Technology}. Chapter 1: ``The Human Capital Century.'' Belknap.
		\item[$\bullet$] Harding, R. and D. Stasavage (2014). ``What Democracy Does (and Doesn’t Do) for Basic Services: School Fees, School Inputs, and African Elections.'' \emph{Journal of Politics}, 76: 229-45.
	\end{itemize}


\item {\bf Inequality}
	\begin{itemize}
		\item[$\bullet$] Piketty, T. (2014). \emph{Capital in the Twenty-First Century}. ``Introduction''. Harvard.
		\item[$\bullet$] Alesina, A., E. Glaeser, and B. Sacerdote (2001). ``Why Doesn't the United States Have a European-Style Welfare State?'' \emph{Brookings Papers on Economic Activity} 2: 187-277. 
		\item[$\bullet$] Scheve, K. and D. Stasavage (2012). ``Democracy, War, and Wealth: Lessons from Two Centuries of Inheritance Taxation.'' \emph{American Political Science Review} 106: 81- 102. 
		\item[$\bullet$] Williamson, J. (2015). ``Latin American Inequality: Colonial Origins, Commodity Booms, or a Missed 20th Century Leveling?'' \emph{National Bureau of Economic Research}, Working Paper 20915.
	\end{itemize}


\item {\bf Culture}
	\begin{itemize}
		\item[$\bullet$] Tabellini, G. (2008). ``Institutions and Culture.'' \emph{Journal of the European Economic
		Association}, 6: 255-294.
		\item[$\bullet$] Giuliano, P. (2015). ``The Role of Women in Society from Preindustrial to Modern
		Times.'' \emph{CESifo Economic Studies}, 61: 33-52.
		\item[$\bullet$] Voigtlader, N. and Voth, H.J. (2015). ``Nazi Indoctrination and Anti-Semitic Beliefs in Germany.'' \emph{Proceedings of the National Academy of Sciences}, 112: 7931-7936.
		\item[$\bullet$] Gladwell, M. (2011). \emph{Outliers}. Chapter 8: ``Rice Paddies and Math Tests.'' Back Bay.
	\end{itemize}



\end{enumerate}




  


  










%\bibliographystyle{plainnat}
%\bibliography{/Users/hectorbahamonde/RU/Bibliografia_PoliSci/Bahamonde_BibTex2013}

\end{document}